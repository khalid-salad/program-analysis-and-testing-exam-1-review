\documentclass{article}
\usepackage{main}
\title{Program Analysis and Testing\\Exam 1 Review}
\author{Khalid Hourani}
\begin{document}
\maketitle
\section{Python for PAT} 
\subsection{Inspecting objects in python}
Python objects can be inspected with a handful of built-in functions.
\begin{table}
\begin{tabular}{l p{10cm}}
  function & description\\\toprule
  \mintinline{python}|help| & Invoke the built-in help system\\\cmidrule{2-2}
  \mintinline{python}|type| & With one argument, return the type of an object\\\cmidrule{2-2}
  \mintinline{python}|dir| & Without arguments, return the list of names in the current local scope. With an argument, attempt to return a list of valid attributes for that object\\\cmidrule{2-2}
  \mintinline{python}|id| & Return the “identity” of an object. This is an integer which is guaranteed to be unique and constant for this object during its lifetime.\\\cmidrule{2-2}
  \mintinline{python}|getattr| & Return the value of the named attribute of object\\\cmidrule{2-2}
  \mintinline{python}|callable| & Return \mintinline{python}|True| if the object argument appears callable, \mintinline{python}|False| if not.
\end{tabular}
\end{table}

\subsection{Magic functions}
A \emph{Magic Method} is a function (always beginning and ending with \texttt{\textunderscore\textunderscore},
called a \emph{dunderstore}). Examples are given in \cref{tbl:magic}.

\begin{table}
\begin{tabular}{l p{10cm}}
  function & description\\\toprule
  \mintinline{python}|__new__| & Called to create a new instance of class cls.\\\cmidrule{2-2}
  \mintinline{python}|__init__| & Called after the instance has been created (by \mintinline{python}|__new__()|), but before it is returned to the caller.\\\cmidrule{2-2}
  \mintinline{python}|__del__| & Called when the instance is about to be destroyed.\\\cmidrule{2-2}
  \mintinline{python}|__repr__| & Called by the \mintinline{python}|repr()| built-in function to compute the ``official'' string representation of an object\\\cmidrule{2-2}
  \mintinline{python}|__str__| & Called by \mintinline{python}|str(object)| and the built-in functions \mintinline{python}|format()| and \mintinline{python}|print()| to compute the ``informal'' or nicely printable string representation of an object.
\end{tabular}
\end{table}
\subsection{Syntactic sugar}
\emph{Syntactic Sugar} is syntax within a programming language that is designed to make things
easier to read or to express. For example, a function decorator can be used as shorthand
for function composition:
\begin{minted}[linenos]{python}
@decorator
def func():
    # do whatever
\end{minted}
is equivalent to
\begin{minted}[linenos]{python}
def func(args):
  # do whatever
func = decorator(func)
\end{minted}
Other examples of syntactic sugar:

Compound inequalities:
\begin{multicols}{2}
\begin{minted}[linenos]{python}
1 < x < 10
\end{minted}
\begin{minted}[linenos]{python}
1 < x and x < 10
\end{minted}
\end{multicols}
List comprehension:
\begin{multicols}{2}
\begin{minted}[linenos]{python}
arr = [x for x in range(10)]
\end{minted}
\vfill\null
\columnbreak
\begin{minted}[linenos]{python}
arr = []
for x in range(10):
    arr.append(x)
\end{minted}
\end{multicols}
\subsection{Regular expression}
A \emph{regular expression} is a sequence of characters that specify a search pattern
in text.
\begin{minted}[linenos]{python}
re.findall(r'\bf[a-z]*', 'which foot or hand fell fastest')
\end{minted}
\Cref{tbl:regex} gives an outline of regular expression syntax.
\begin{table}
\begin{tabular}{l p{10cm}}
expression & explanation\\\toprule
\mintinline{python}|.| & (Dot.) In the default mode, this matches any character except a newline. If the \mintinline{python}|DOTALL| flag has been specified, this matches any character including a newline.\\\cmidrule{2-2}
\mintinline{python}|^| & (Caret.) Matches the start of the string, and in \mintinline{python}|MULTILINE| mode also matches immediately after each newline.\\\cmidrule{2-2}
\texttt{\$} & Matches the end of the string or just before the newline at the end of the string. 
\end{tabular}
\end{table} 
\section{Concepts and application of concepts in PAT} 
\subsection{Program Concrete/Abstract/Symbolic State} 
\subsection{State space} 
\subsection{Overapproximation} 
\subsection{Reachability} 
\subsection{Safety and Liveness properties} 
\subsection{Meta-morphic relations} 
\subsection{Undecidablity} 
\subsection{Satisifiability} 

\section{Control flow graph} 
\subsection{Basic blocks} 
\subsection{Transitions} 
 
\section{Data flow} 
\subsection{Def/Use} 
\subsection{Def-use pairs} 
\subsection{Def/Use in presence of references} 
\subsection{Data flow algorithms}
\end{document}