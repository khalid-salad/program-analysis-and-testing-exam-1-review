\documentclass{article}
\usepackage{main}
\title{Program Analysis and Testing\\Exam 1 Review}
\author{Khalid Hourani}
\begin{document}
\maketitle
\section{Python for PAT} 
\subsection{Inspecting objects in python}
Python objects can be inspected with a handful of built-in functions.
\begin{table}[H]
\begin{tabular}{l p{10cm}}
  function & description\\\toprule
  \mintinline{python}|help| & Invoke the built-in help system\\\cmidrule{2-2}
  \mintinline{python}|type| & With one argument, return the type of an object\\\cmidrule{2-2}
  \mintinline{python}|dir| & Without arguments, return the list of names in the current local scope. With an argument, attempt to return a list of valid attributes for that object\\\cmidrule{2-2}
  \mintinline{python}|id| & Return the “identity” of an object. This is an integer which is guaranteed to be unique and constant for this object during its lifetime.\\\cmidrule{2-2}
  \mintinline{python}|getattr| & Return the value of the named attribute of object\\\cmidrule{2-2}
  \mintinline{python}|callable| & Return \mintinline{python}|True| if the object argument appears callable, \mintinline{python}|False| if not.
\end{tabular}
\end{table}

\subsection{Magic functions}
A \emph{Magic Method} is a function (always beginning and ending with \texttt{\textunderscore\textunderscore},
called a \emph{dunderstore}). Examples are given in \cref{tbl:magic}.

\begin{table}[H]
\begin{tabular}{l p{10cm}}
  function & description\\\toprule
  \mintinline{python}|__new__| & Called to create a new instance of class cls.\\\cmidrule{2-2}
  \mintinline{python}|__init__| & Called after the instance has been created (by \mintinline{python}|__new__()|), but before it is returned to the caller.\\\cmidrule{2-2}
  \mintinline{python}|__del__| & Called when the instance is about to be destroyed.\\\cmidrule{2-2}
  \mintinline{python}|__repr__| & Called by the \mintinline{python}|repr()| built-in function to compute the ``official'' string representation of an object\\\cmidrule{2-2}
  \mintinline{python}|__str__| & Called by \mintinline{python}|str(object)| and the built-in functions \mintinline{python}|format()| and \mintinline{python}|print()| to compute the ``informal'' or nicely printable string representation of an object.
\end{tabular}
\end{table}
\subsection{Syntactic sugar}
\emph{Syntactic Sugar} is syntax within a programming language that is designed to make things
easier to read or to express. For example, a function decorator can be used as shorthand
for function composition:
\begin{minted}[linenos]{python}
@decorator
def func():
    # do whatever
\end{minted}
is equivalent to
\begin{minted}[linenos]{python}
def func(args):
  # do whatever
func = decorator(func)
\end{minted}
Other examples of syntactic sugar:

Compound inequalities:
\begin{multicols}{2}
\begin{minted}[linenos]{python}
1 < x < 10
\end{minted}
\begin{minted}[linenos]{python}
1 < x and x < 10
\end{minted}
\end{multicols}
List comprehension:
\begin{multicols}{2}
\begin{minted}[linenos]{python}
arr = [x for x in range(10)]
\end{minted}
\vfill\null
\columnbreak
\begin{minted}[linenos]{python}
arr = []
for x in range(10):
    arr.append(x)
\end{minted}
\end{multicols}
\subsection{Regular expression}
A \emph{regular expression} is a sequence of characters that specify a search pattern
in text.
\begin{minted}[linenos]{python}
re.findall(r'\bf[a-z]*', 'which foot or hand fell fastest')
\end{minted}
\Cref{tbl:regex} gives an outline of regular expression syntax.
\begin{table}[H]
\begin{tabular}{l p{10cm}}
expression & explanation\\\toprule
\mintinline{python}|.| & (Dot.) In the default mode, this matches any character except a newline. If the \mintinline{python}|DOTALL| flag has been specified, this matches any character including a newline.\\\cmidrule{2-2}
\mintinline{python}|^| & (Caret.) Matches the start of the string, and in \mintinline{python}|MULTILINE| mode also matches immediately after each newline.\\\cmidrule{2-2}
\texttt{\$} & Matches the end of the string or just before the newline at the end of the string.\\\cmidrule{2-2}
\mintinline{python}|*| & Causes the resulting RE to match 0 or more repetitions of the preceding RE\\\cmidrule{2-2}
\texttt{+} & Causes the resulting RE to match 1 or more repetitions of the preceding RE\\\cmidrule{2-2}
\texttt{?} & Causes the resulting RE to match 0 or 1 repetitions of the preceding RE.\\\cmidrule{2-2}
\texttt{[]} & Used to indicate a set of characters
\end{tabular}
\end{table} 
\section{Concepts and application of concepts in PAT} 
\subsection{Program Concrete/Abstract/Symbolic State} 
\subsection{State space} 
\subsection{Overapproximation} 
\subsection{Reachability} 
\subsection{Safety and Liveness properties} 
\subsection{Meta-morphic relations} 
\subsection{Undecidablity} 
\subsection{Satisifiability} 

\section{Control flow graph} 
A \emph{control flow graph} is a representation, using graph notation, of all paths that might be traversed through a program during its execution.
\subsection{Basic blocks}
The \emph{nodes} in the graph correspond to regions of source code. A \emph{basic block} is maximal program region with a single entry and single exit point.
\subsection{Transitions} 
The (directed) \emph{edges} of the graph correspond to the possibility that program execution proceeds from the end of one region directly to the beginning of another.
 
\section{Data flow}
\subsection{Def/Use}
Definition: where a variable gets a value
\begin{itemize}[nosep]
    \item Variable declaration  (often the special value “uninitialized”)
    \item Variable initialization
    \item Assignment
    \item Values received by a parameter 
\end{itemize}
Use: extraction of a value from a variable
\begin{itemize}[nosep]
    \item Expressions
    \item Conditional statements
    \item Parameter passing
    \item Returns
\end{itemize}
\subsection{Def-use pairs} 
A \emph{def-use} (du) pair associates a point in a program where a value is produced with a point where it is used.

\subsection{Def/Use in presence of references}
\subsection{Data flow algorithms}
Suppose we are calculating the reaching definitions of node $v$, and there is an edge $(p, v)$ from an immediate predecessor node $p$.
\begin{itemize}[nosep]
    \item If the predecessor node $p$ can assign a value to variable $x$, then the definition $x_p$ reaches $v$.  We say the definition $x_p$ is generated at $p$.
    \item If a definition $x_p$ of variable $x$ reaches a predecessor node $p$, and if $x$ is not redefined at that node (in which case we say the $x_p$ is killed
	  at that point), then the definition is propagated on from $p$ to $v$.
\end{itemize}

Worklist algorithm iterate to a fixed point solution.

General idea: 
\begin{itemize}
    \item Initially all nodes are on the work list, and have default values 
    \item Default for ``any-path'' problem is the empty set, default for ``all-path'' problem is the set of all possibilities (union of all gen sets)
    \item While the work list is not empty
    \begin{itemize}
	    \item Pick any node n on work list; remove it from the list
        \item Apply the data flow equations for that node to get new values
        \item If the new value is changed (from the old value at that node), then 
        \begin{itemize}
		    \item Add successors (for forward analysis) or predecessors (for backward analysis) on the work list
		\end{itemize}
	\end{itemize}
    \item Eventually the work list will be empty (because new computed values = old values for each node) and the algorithm stops.
\end{itemize}
\end{document}